%% Generated by Sphinx.
\def\sphinxdocclass{report}
\documentclass[letterpaper,10pt,english]{sphinxmanual}
\ifdefined\pdfpxdimen
   \let\sphinxpxdimen\pdfpxdimen\else\newdimen\sphinxpxdimen
\fi \sphinxpxdimen=.75bp\relax

\PassOptionsToPackage{warn}{textcomp}
\usepackage[utf8]{inputenc}
\ifdefined\DeclareUnicodeCharacter
% support both utf8 and utf8x syntaxes
\edef\sphinxdqmaybe{\ifdefined\DeclareUnicodeCharacterAsOptional\string"\fi}
  \DeclareUnicodeCharacter{\sphinxdqmaybe00A0}{\nobreakspace}
  \DeclareUnicodeCharacter{\sphinxdqmaybe2500}{\sphinxunichar{2500}}
  \DeclareUnicodeCharacter{\sphinxdqmaybe2502}{\sphinxunichar{2502}}
  \DeclareUnicodeCharacter{\sphinxdqmaybe2514}{\sphinxunichar{2514}}
  \DeclareUnicodeCharacter{\sphinxdqmaybe251C}{\sphinxunichar{251C}}
  \DeclareUnicodeCharacter{\sphinxdqmaybe2572}{\textbackslash}
\fi
\usepackage{cmap}
\usepackage[T1]{fontenc}
\usepackage{amsmath,amssymb,amstext}
\usepackage{babel}
\usepackage{times}
\usepackage[Bjarne]{fncychap}
\usepackage{sphinx}

\fvset{fontsize=\small}
\usepackage{geometry}

% Include hyperref last.
\usepackage{hyperref}
% Fix anchor placement for figures with captions.
\usepackage{hypcap}% it must be loaded after hyperref.
% Set up styles of URL: it should be placed after hyperref.
\urlstyle{same}
\addto\captionsenglish{\renewcommand{\contentsname}{Contents:}}

\addto\captionsenglish{\renewcommand{\figurename}{Fig.\@ }}
\makeatletter
\def\fnum@figure{\figurename\thefigure{}}
\makeatother
\addto\captionsenglish{\renewcommand{\tablename}{Table }}
\makeatletter
\def\fnum@table{\tablename\thetable{}}
\makeatother
\addto\captionsenglish{\renewcommand{\literalblockname}{Listing}}

\addto\captionsenglish{\renewcommand{\literalblockcontinuedname}{continued from previous page}}
\addto\captionsenglish{\renewcommand{\literalblockcontinuesname}{continues on next page}}
\addto\captionsenglish{\renewcommand{\sphinxnonalphabeticalgroupname}{Non-alphabetical}}
\addto\captionsenglish{\renewcommand{\sphinxsymbolsname}{Symbols}}
\addto\captionsenglish{\renewcommand{\sphinxnumbersname}{Numbers}}

\addto\extrasenglish{\def\pageautorefname{page}}

\setcounter{tocdepth}{1}



\title{QGIS\_documentation Documentation}
\date{Jul 03, 2020}
\release{}
\author{Lorenzo Amici}
\newcommand{\sphinxlogo}{\vbox{}}
\renewcommand{\releasename}{}
\makeindex
\begin{document}

\pagestyle{empty}
\sphinxmaketitle
\pagestyle{plain}
\sphinxtableofcontents
\pagestyle{normal}
\phantomsection\label{\detokenize{index::doc}}



\chapter{Title}
\label{\detokenize{index:title}}

\section{Subtitle}
\label{\detokenize{index:subtitle}}
\sphinxurl{https://docs.qgis.org/3.10/en/docs/index.html}

\sphinxhref{https://docs.qgis.org/3.10/en/docs/index.html}{QGIS}


\subsection{Sub subtitle}
\label{\detokenize{index:sub-subtitle}}
\sphinxguilabel{Processing Toolbox}

\sphinxcode{\sphinxupquote{excercise data}}

\noindent{\hspace*{\fill}\sphinxincludegraphics[width=200\sphinxpxdimen]{{1200px-QGIS_logo,_2017.svg}.png}\hspace*{\fill}}

\begin{figure}[htbp]
\centering
\capstart

\noindent\sphinxincludegraphics[width=200\sphinxpxdimen]{{1200px-QGIS_logo,_2017.svg}.png}
\caption{QGIS logo}\label{\detokenize{index:id1}}\end{figure}


\sphinxstrong{See also:}


This is a simple \sphinxstylestrong{seealso} note.



\begin{sphinxadmonition}{note}{Note:}
This is a \sphinxstylestrong{note} box.
\end{sphinxadmonition}

\begin{sphinxadmonition}{warning}{Warning:}
This is a \sphinxstyleemphasis{warning}.
\end{sphinxadmonition}


\subsubsection{Sub sub subtitle}
\label{\detokenize{index:sub-sub-subtitle}}

\paragraph{General Info}
\label{\detokenize{introduction/general:general-info}}\label{\detokenize{introduction/general::doc}}
The OSGeo UN Committee promotes the development and use of open source software that meets UN needs and supports the aims of the UN. Following a meeting between OSGeo Board of Directors and the UN GIS team at FOSS4G in Seoul, Korea in September 2015, the Committee has mainly worked on the UN Open GIS Initiative, a project “…to identify and develop an Open Source GIS bundle that meets the requirements of UN operations, taking full advantage of the expertise of mission partners including partner nations, technology contributing countries, international organisations, academia, NGOs, private sector. The strategic approach shall be developed with best and shared principles, standards and ownership in a prioritized manner that addresses capability gaps and needs without duplicating efforts of other Member States or entities. The UN Open GIS Initiative strategy shall collaboratively and cooperatively develop, validate, assess, migrate and implement sound technical capabilities with all the appropriate documentation and training that in the end provides a united effort to improve the effectiveness and efficiency of utilizing Open Source GIS around the world.”


\subparagraph{1. Purpose of this documentation}
\label{\detokenize{introduction/general:purpose-of-this-documentation}}
This educational material is designed as a step-by-step software learning guide for QGIS.


\subsection{Indices and tables}
\label{\detokenize{index:indices-and-tables}}\begin{itemize}
\item {} 
\DUrole{xref,std,std-ref}{genindex}

\item {} 
\DUrole{xref,std,std-ref}{modindex}

\item {} 
\DUrole{xref,std,std-ref}{search}

\end{itemize}



\renewcommand{\indexname}{Index}
\printindex
\end{document}